\documentclass[aoas]{imsart}\usepackage[]{graphicx}\usepackage[]{color}
%% maxwidth is the original width if it is less than linewidth
%% otherwise use linewidth (to make sure the graphics do not exceed the margin)
\makeatletter
\def\maxwidth{ %
  \ifdim\Gin@nat@width>\linewidth
    \linewidth
  \else
    \Gin@nat@width
  \fi
}
\makeatother

\definecolor{fgcolor}{rgb}{0.345, 0.345, 0.345}
\newcommand{\hlnum}[1]{\textcolor[rgb]{0.686,0.059,0.569}{#1}}%
\newcommand{\hlstr}[1]{\textcolor[rgb]{0.192,0.494,0.8}{#1}}%
\newcommand{\hlcom}[1]{\textcolor[rgb]{0.678,0.584,0.686}{\textit{#1}}}%
\newcommand{\hlopt}[1]{\textcolor[rgb]{0,0,0}{#1}}%
\newcommand{\hlstd}[1]{\textcolor[rgb]{0.345,0.345,0.345}{#1}}%
\newcommand{\hlkwa}[1]{\textcolor[rgb]{0.161,0.373,0.58}{\textbf{#1}}}%
\newcommand{\hlkwb}[1]{\textcolor[rgb]{0.69,0.353,0.396}{#1}}%
\newcommand{\hlkwc}[1]{\textcolor[rgb]{0.333,0.667,0.333}{#1}}%
\newcommand{\hlkwd}[1]{\textcolor[rgb]{0.737,0.353,0.396}{\textbf{#1}}}%
\let\hlipl\hlkwb

\usepackage{framed}
\makeatletter
\newenvironment{kframe}{%
 \def\at@end@of@kframe{}%
 \ifinner\ifhmode%
  \def\at@end@of@kframe{\end{minipage}}%
  \begin{minipage}{\columnwidth}%
 \fi\fi%
 \def\FrameCommand##1{\hskip\@totalleftmargin \hskip-\fboxsep
 \colorbox{shadecolor}{##1}\hskip-\fboxsep
     % There is no \\@totalrightmargin, so:
     \hskip-\linewidth \hskip-\@totalleftmargin \hskip\columnwidth}%
 \MakeFramed {\advance\hsize-\width
   \@totalleftmargin\z@ \linewidth\hsize
   \@setminipage}}%
 {\par\unskip\endMakeFramed%
 \at@end@of@kframe}
\makeatother

\definecolor{shadecolor}{rgb}{.97, .97, .97}
\definecolor{messagecolor}{rgb}{0, 0, 0}
\definecolor{warningcolor}{rgb}{1, 0, 1}
\definecolor{errorcolor}{rgb}{1, 0, 0}
\newenvironment{knitrout}{}{} % an empty environment to be redefined in TeX

\usepackage{alltt}
%\usepackage{setspace}

\usepackage{dsfont}
\usepackage{amsthm,amsmath,amssymb,natbib}
\RequirePackage[colorlinks,citecolor=blue,urlcolor=blue]{hyperref}
\usepackage{xspace,soul}
\usepackage{graphicx}

\usepackage[margin=1.45in]{geometry}

\startlocaldefs
\newcommand{\blind}{Three anonymous authors}

\newcommand{\Ex}{\mathbb{E}}
\newcommand{\Var}{\text{Var}}
\newcommand{\bp}{\mathbf{p}}

\newcommand{\R}{\textsf{R}\xspace}
\newcommand{\pkg}[1]{\texttt{#1}\xspace}

\newcommand{\greg}[1]{\sethlcolor{yellow}\hl{[GM]: #1}}
\newcommand{\ben}[1]{\sethlcolor{green}\hl{[BB]: #1}}
\newcommand{\mike}[1]{\sethlcolor{cyan}\hl{[ML]: #1}}

\def\balpha{\pmb{\alpha}}
\def\btheta{\pmb{\theta}}
\def\bgamma{\pmb{\gamma}}
\def\btheta{\pmb{\theta}}
\def\bphi{\pmb{\phi}}
\def\bpsi{\pmb{\psi}}
\def\bB{\pmb{B}}
\def\bD{\pmb{D}}
\def\bH{\pmb{H}}
\def\bS{\pmb{S}}
\def\bX{\pmb{X}}

\endlocaldefs
\IfFileExists{upquote.sty}{\usepackage{upquote}}{}
\begin{document}

\begin{frontmatter}

\title{Rush Independent Passing Player Efficiency Number (RIPPEN)}
\runtitle{RIPPEN}



\author{\fnms{Gregory J.} \snm{Matthews}\corref{}\ead[label=e1]{gmatthews1@luc.edu}}
\address{\printead{e1}}
\affiliation{Skidmore College}

\and
\author{\fnms{Russell} \snm{Cain}\ead[label=e2]{rcain@luc.edu}}
\address{\printead{e2}}
\affiliation{Loyola University Chicago}

\and
\author{\fnms{Donald} \snm{Stolz}\ead[label=e3]{dstolz@luc.edu}}
\address{\printead{e3}}
\affiliation{Smith College}


\runauthor{Lopez, Matthews, Baumer}

\begin{abstract}
RIPPEN, Rush Independent Passing Player Efficiency Number, is a new measurement of passer performance. In a simulated world, how would a passer perform starting from their twenty yard line and only performing pass plays? The aspects of each play are simulated using a Bayesian model. This allows rookies and backups with minimal data to be fairly evaluated. Drives would end in a touchdown, field goal or turnover. A player’s RIPPEN is the average number of points they would be expected to score per drive. Our metric improves on existing passer rating systems because it is updated to current NFL data, does not weight passing touchdowns, and it is able to be more intuitively understood.

\end{abstract}

\begin{keyword}
\kwd{sports analytics}
\kwd{Bayesian modeling}
\kwd{competitive balance}
\kwd{MCMC}
\end{keyword}

\end{frontmatter}
%%%%%%%%%%%%%%%%%%%%%%%%%%%%%%%%%%%%%%%%%%%%%%%%%%%%%%%%%%%%%%%%%%%%%%%

%\input{intro}

%Censoring in rjags
\section{Introduction}




The current passer rating measure has been around since 1973.  

NFL's Quarterback Rating: 

Using the notation from \cite{vanDohlen2011}:
$$
QBR = \left(\frac{\frac{C}{A}-0.3}{0.2} + \frac{\frac{Y}{A}-3}{4} + \frac{\frac{T}{A}}{0.05} + \frac{0.095-\frac{I}{A}}{0.04}\right)\left(\frac{100}{6}\right)
$$
where 
$C$ = Number of Completions\\
$Y$ = Number of Yards\\
$A$ = Number of Attempts\\
$T$ = Number of Touchdowns\\
$I$ = Number of Interceptions\\

Each of these four components has a maximum of 2.375, so a ``perfect" passer rating in the NFL is $\frac{(2.375)(4)(100)}{6} = 158.3$.  

The NCAA passer rating is: 
$\frac{8.4Y + 330T + 100C - 200I}{A}$

This number ranged from -731.6 to 1261.6.  Ridiculous.  


Passer rating is bad.  RIPPEN is better.  
The NCAA and NFL formulas are different. Mention this.  

Tim Tebow example of why QBR is bad: 
%http://bleacherreport.com/articles/889199-big-flaws-in-espns-total-qbr-exposed-after-tim-tebow-rates-above-aaron-rodgers#

Read more about this.  Might be interesting:  
%https://412sportsanalytics.wordpress.com/2016/10/05/pareto-frontier-and-non-dominated-quarterbacks/
- Would we add something like this to our results

DYAR and DVOA: %https://www.footballoutsiders.com/stats/qb


http://www.nih.ticz.musclehedz.charlespoliquin.sportsci.org/2011/mep.htm


JQAS: https://www.degruyter.com/view/j/jqas.2011.7.3/jqas.2011.7.3.1359/jqas.2011.7.3.1359.xml?format=INT

A Statistical Analysis of NFL Quarterback Rating Variables
Derek Stimel, Journal of Quantitative Analysis in Sports
The Quarterback Prediction Problem: Forecasting the Performance of College Quarterbacks Selected in the NFL Draft
Julian Wolfson et al., Journal of Quantitative Analysis in Sports
Analyzing dependence matrices to investigate relationships between national football league combine event performances
Brook T. Russell et al., Journal of Quantitative Analysis in Sports
Isolating the Effect of Individual Linemen on the Passing Game in the National Football League
Benjamin C Alamar et al., Journal of Quantitative Analysis in Sports
Quantifying NFL Coaching: A Proof of New Growth Theory
Kevin P. Braig, Journal of Quantitative Analysis in Sports


CITE Passer Rating
CITE QBR

Don Steinberg: How I Learned to Stop Worrying and Love the Bomb
http://www.donsteinberg.com/qbrating.htm

Quarterback Rating: 
http://www.nfl.com/help/quarterbackratingformula

NFL Passer rating: 
https://www.profootballhof.com/news/nfl-s-passer-rating/

College Passer efficiency: 
http://football.stassen.com/pass-eff/

Defending Passer rating: Kerry Byrne
https://www.si.com/more-sports/2011/08/03/defending-qb-rating

PRO FOOTBALL; The N.F.L.'s Passer Rating, Arcane and Misunderstood
https://www.nytimes.com/2004/01/14/sports/pro-football-the-nfl-s-passer-rating-arcane-and-misunderstood.html

\cite{Stimel2009}
Looking for structural breaks in QBR.  

\cite{vanDohlen2011}

\subsection{Criticism of QBR}
Arbitrary scale (0 to 158.3??)
Hard to interpret (What does 121.6 mean?)
QBR overly credits QBs for scoring TDs.  





\section{Methods}
We propose Rush Independent Passing Player Efficiency Number (RIPPEN).  
Describe what we did.  

RIPPEN uses the exact same inputs as QBR: 
Completions, Yards, Interceptions, and Touchdowns.  

 

\subsection{Theoretcal Results}
Do we have any? 


\subsection{Correlation between RIPPEN and winning}
Compare RIPPEN and winning to QBR and winning.  

\subsection{Preliminary Results \& Notes}

This will house all of amusing musings from the past few weeks while we wait for the overall paper to take form. First thing's first, we need to know what if left to be done:

\subsubsection{TO DO:}
\begin{itemize}
	\item Type up Markov Chain
	\item Implement %\href{http://mcmc-jags.sourceforge.net/}{JAGS?}
	\item Censored Data
\end{itemize}

\subsubsection{Markov Chain Notion: }
\begin{center}
	\begin{tabular}{|c |c c c c|} 
		\hline
		-Markov- & Down 1 & Down 2 & Down 3 & Down 4 \\ \hline
		Down 1 & a & b & 0 & 0 \\ 
		\hline
		Down 2 & c & 0 & d & 0 \\
		\hline
		Down 3 & e & 0 & 0 & f\\
		\hline
		Down 4 & 0 & 0 & 0 & 1 \\
		\hline
	\end{tabular}
	\begin{align*}
	a &= Pr(y_{d,1} > 10)\\
	b &= 1-a\\
	c &= Pr(y_{d,2} > 10 - y_{d,1})\\
	d &= 1 - c\\
	e &= Pr(y_{d,3} > 10 - y_{d,2} - y_{d,1})\\
	f &= 1 - e
	\end{align*}
\end{center}

\subsubsection{Closed Form? Here is all of the Notation:}
Our notation becomes rather hairy around here, so bear with me. I will just go ahead and write down the equations we made up and try to tell which of the 14 different i's imply what: 


%Something in here is not compiling.
% \begin{enumerate}[(a)]
% 	\item $y_{d,i}$: The $i^{th}$ pass of the $d^{th}$ down series. Therefore, $i \in {1,2,3,4}$ and $d$ is loosely less than 8. 
% 	\item G: The result of the drive/simulation. Either 7 for TD, 3 for FG or 0 for interception or missed FG.
% 	\item I(...): Indicator function: ... 
% 	\item $C_{d,i}$ ... 
% 	\item $I_{d,i}$: E[ I(D = 4) ] = P(D = 4)
% 	\item $C_{d,i}$: $t'_{1} \cdot M$ = $t'_{2}$ = [$a$  $b$  0  0]
% 	\item .... : $t'_{2} \cdot M$ = $t'_{3}$ = [$a^{2}+bc$  $ab$  $bd$  0]
% 	\item $$Pr(G_{j} = 3) = Pr(FG \cap (\sum_{i=1}^{n-1} I(D_{n} = 4) = 0 ) \cap (\sum_{i=1}^{n-1} y_{i} < 80 | Q = \sum_{i=1}^{n-1} y_{i})) \cdot P(Q = q)$$ \\... Pr(FG $\cap$ Q = q) ?
% 	\item $$Pr(G=7) = \sum_{n=1}^{\infty} Pr(\sum_{i = 1}^{n}y_{i} > 80 | \sum_{i=1}^{n}I(D_{i} = 4) = 0) \cdot P("...")$$
% 	
% 	
% \end{enumerate}



\section{Results}
\subsection{Bayesian Posterior Distributions Stuff}
What do the posterior parameters look like? 

\subsection{Rodgers vs Tebow Example. }

\subsection{Distribtuion of RIPPEN}

\subsection{Best Games/Seasons}

\section{Conclusion and Future Work}
RIPPEN is good.  We will do more eventually.  

Adding a defensive adjustment.  


Do we even want to add these things?  
How do we deal with pass interference.  
Defensive Holding? 
Sacks? Add another layer.  
Fumbles? Could treat similar to interceptions? 
Should interceptions ever result in negative numbers?  
How do we assign the negative numbers for interceptions?  








\bibliographystyle{imsart-nameyear}
\bibliography{refs}




\end{document}
