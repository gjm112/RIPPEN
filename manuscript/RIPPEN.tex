\documentclass[aoas]{imsart}\usepackage[]{graphicx}\usepackage[]{xcolor}
% maxwidth is the original width if it is less than linewidth
% otherwise use linewidth (to make sure the graphics do not exceed the margin)
\makeatletter
\def\maxwidth{ %
  \ifdim\Gin@nat@width>\linewidth
    \linewidth
  \else
    \Gin@nat@width
  \fi
}
\makeatother

\definecolor{fgcolor}{rgb}{0.345, 0.345, 0.345}
\newcommand{\hlnum}[1]{\textcolor[rgb]{0.686,0.059,0.569}{#1}}%
\newcommand{\hlsng}[1]{\textcolor[rgb]{0.192,0.494,0.8}{#1}}%
\newcommand{\hlcom}[1]{\textcolor[rgb]{0.678,0.584,0.686}{\textit{#1}}}%
\newcommand{\hlopt}[1]{\textcolor[rgb]{0,0,0}{#1}}%
\newcommand{\hldef}[1]{\textcolor[rgb]{0.345,0.345,0.345}{#1}}%
\newcommand{\hlkwa}[1]{\textcolor[rgb]{0.161,0.373,0.58}{\textbf{#1}}}%
\newcommand{\hlkwb}[1]{\textcolor[rgb]{0.69,0.353,0.396}{#1}}%
\newcommand{\hlkwc}[1]{\textcolor[rgb]{0.333,0.667,0.333}{#1}}%
\newcommand{\hlkwd}[1]{\textcolor[rgb]{0.737,0.353,0.396}{\textbf{#1}}}%
\let\hlipl\hlkwb

\usepackage{framed}
\makeatletter
\newenvironment{kframe}{%
 \def\at@end@of@kframe{}%
 \ifinner\ifhmode%
  \def\at@end@of@kframe{\end{minipage}}%
  \begin{minipage}{\columnwidth}%
 \fi\fi%
 \def\FrameCommand##1{\hskip\@totalleftmargin \hskip-\fboxsep
 \colorbox{shadecolor}{##1}\hskip-\fboxsep
     % There is no \\@totalrightmargin, so:
     \hskip-\linewidth \hskip-\@totalleftmargin \hskip\columnwidth}%
 \MakeFramed {\advance\hsize-\width
   \@totalleftmargin\z@ \linewidth\hsize
   \@setminipage}}%
 {\par\unskip\endMakeFramed%
 \at@end@of@kframe}
\makeatother

\definecolor{shadecolor}{rgb}{.97, .97, .97}
\definecolor{messagecolor}{rgb}{0, 0, 0}
\definecolor{warningcolor}{rgb}{1, 0, 1}
\definecolor{errorcolor}{rgb}{1, 0, 0}
\newenvironment{knitrout}{}{} % an empty environment to be redefined in TeX

\usepackage{alltt}
%\usepackage{setspace}

\usepackage{dsfont}
\usepackage{amsthm,amsmath,amssymb,natbib}
\RequirePackage[colorlinks,citecolor=blue,urlcolor=blue]{hyperref}
\usepackage{xspace,soul}
\usepackage{graphicx}

\usepackage[margin=1.45in]{geometry}

\startlocaldefs
\newcommand{\blind}{Three anonymous authors}

\newcommand{\Ex}{\mathbb{E}}
\newcommand{\Var}{\text{Var}}
\newcommand{\bp}{\mathbf{p}}

\newcommand{\R}{\textsf{R}\xspace}
\newcommand{\pkg}[1]{\texttt{#1}\xspace}

\newcommand{\greg}[1]{\sethlcolor{yellow}\hl{[GM]: #1}}
\newcommand{\ben}[1]{\sethlcolor{green}\hl{[BB]: #1}}
\newcommand{\mike}[1]{\sethlcolor{cyan}\hl{[ML]: #1}}

\def\balpha{\pmb{\alpha}}
\def\btheta{\pmb{\theta}}
\def\bgamma{\pmb{\gamma}}
\def\btheta{\pmb{\theta}}
\def\bphi{\pmb{\phi}}
\def\bpsi{\pmb{\psi}}
\def\bB{\pmb{B}}
\def\bD{\pmb{D}}
\def\bH{\pmb{H}}
\def\bS{\pmb{S}}
\def\bX{\pmb{X}}

\endlocaldefs
\IfFileExists{upquote.sty}{\usepackage{upquote}}{}
\begin{document}
\SweaveOpts{concordance=TRUE}


\begin{frontmatter}

\title{Rush Independent Passing Player Efficiency Number (RIPPEN)}
\runtitle{RIPPEN}



\author{\fnms{Gregory J.} \snm{Matthews}\corref{}\ead[label=e1]{gmatthews1@luc.edu}}
\address{\printead{e1}}
\affiliation{Skidmore College}

\and
\author{\fnms{Russell} \snm{Cain}\ead[label=e2]{rcain@luc.edu}}
\address{\printead{e2}}
\affiliation{Loyola University Chicago}

\and
\author{\fnms{Donald} \snm{Stolz}\ead[label=e3]{dstolz@luc.edu}}
\address{\printead{e3}}
\affiliation{Loyola University Chicago}

 
\runauthor{Stolz, Matthews, Cain
}

\begin{abstract}
RIPPEN, Rush Independent Passing Player Efficiency Number, is a new measurement of passer performance. In a simulated world, how would a passer perform starting from their twenty yard line and only performing pass plays? The aspects of each play are simulated using a Bayesian model. This allows rookies and backups with minimal data to be fairly evaluated. Drives would end in a touchdown, field goal or turnover. A player’s RIPPEN is the average number of points they would be expected to score per game. Our metric improves on existing passer rating systems because it is updated to current NFL data, does not weight situational factors, and it is able to be more intuitively understood.

\end{abstract}

\begin{keyword}
\kwd{sports analytics}
\kwd{Bayesian modeling}
\kwd{competitive balance}
\kwd{MCMC}
\end{keyword}

\end{frontmatter}
%%%%%%%%%%%%%%%%%%%%%%%%%%%%%%%%%%%%%%%%%%%%%%%%%%%%%%%%%%%%%%%%%%%%%%%

%\input{intro}

%Censoring in rjags
\section{Introduction}
Quantifying the performance of a quarterback is one of the most persistent challenges in sports analytics. While numerous metrics exist, most rely on arbitrary weighting systems or proprietary formulas that lack transparency. In this section, we review the most prominent passer rating systems - NFL Passer Rating, NCAA Passing Efficiency, Total QBR, DVOA, and DYAR - discuss their limitations, and then introduce RIPPEN, a new measure of passer performance.

\subsection{NFL Passer Rating}

The National Football League (NFL) passer rating formula has been the official standard since the 1973 season. Developed by a committee headed by Don Smith and Seymour Siwoff, the metric was designed to compare passers against a fixed performance standard based on data from the 1960-1970 era.

The formula relies on four components: completion percentage, yards per attempt, touchdown percentage, and interception percentage. Each component is calculated independently, bounded by a floor of zero and a cap of 2.375, and then combined using the following formula: 


All of this leads to the NFL formula for passer rating (http://www.nfl.com/help/quarterbackratingformula).  Using the notation from \cite{vanDohlen2011}:
$$
QBR = \left(\frac{\frac{C}{A}-0.3}{0.2} + \frac{\frac{Y}{A}-3}{4} + \frac{\frac{T}{A}}{0.05} + \frac{0.095-\frac{I}{A}}{0.04}\right)\left(\frac{100}{6}\right)
$$
where 
$C$ = Number of Completions\\
$Y$ = Number of Yards\\
$A$ = Number of Attempts\\
$T$ = Number of Touchdowns\\
$I$ = Number of Interceptions\\

This calculation results in a rating between 0 and a maximum of 158.3.

\subsection{NCAA Passer Rating}

College football utilizes a similar but distinct formula known as NCAA Passing Efficiency. Like the NFL metric, it uses completions, yards, touchdowns, and interceptions. However, the NCAA formula does not cap individual components, allowing for a much wider variance in scores. The formula is calculated as:

$$
\frac{8.4Y + 330T + 100C - 200I}{A}
$$
where 
$C$ = Number of Completions\\
$Y$ = Number of Yards\\
$A$ = Number of Attempts\\
$T$ = Number of Touchdowns\\
$I$ = Number of Interceptions\\

Because there are no artificial upper limits on the inputs, the theoretical range of the NCAA rating is significantly wider than the NFL’s, ranging from -731.6 to 1,261.6.

\subsection{Total QBR}
In 2011, ESPN introduced the Total Quarterback Rating (Total QBR) to address the limitations of traditional box-score metrics. Unlike the NFL and NCAA ratings, Total QBR attempts to incorporate contextual elements by utilizing Expected Points Added (EPA). This approach evaluates the difference in expected points before and after a play, rewarding quarterbacks for plays that increase the team's probability of scoring. Total QBR also accounts for rushing contributions, sacks, fumbles, and penalties, and includes adjustments for opponent strength and "garbage time" scenarios.

\subsection{DVOA and DYAR}
Football Outsiders introduced two prominent advanced metrics: Defense-adjusted Value Over Average (DVOA) and Defense-adjusted Yards Above Replacement (DYAR). DVOA measures a quarterback's per-play efficiency by comparing each down and distance situation to a league baseline, subsequently adjusting for the strength of the opposing defense. While DVOA acts as a rate statistic expressed as a percentage above or below the league average, DYAR converts this efficiency into a cumulative volume metric, quantifying the total yards a quarterback contributes over a replacement-level alternative.


\subsection{Criticism of Existing Passer Rating Systems}
Despite their widespread use, existing metrics suffer from significant methodological flaws:
\begin{itemize}
\item Arbitrary Weighting and Caps: The NFL Passer Rating utilizes weights derived from 1970s defensive standards, which are increasingly irrelevant in the modern offensive era. Furthermore, the imposition of a 2.375 cap on component statistics is arbitrary; a quarterback who completes 77.5% of their passes receives the same component score as one who completes 90%.
\item Lack of Interpretability: Both the NFL (0–158.3) and NCAA scales are difficult to interpret intuitively. A difference between a rating of 85 and 95 does not translate to a tangible unit of football value (e.g., points or yards).
\item Lack of Transparency: While Total QBR and DVOA attempt to add context, they are largely "black box" metrics. The exact coefficients, replacement-level baselines, and opponent adjustment formulas for DVOA are proprietary. This lack of transparency makes it impossible for independent researchers to verify, reproduce, or fully audit the results.
\item Contextual Blindness: Traditional passer ratings treat all yards equally. A 10-yard completion on 3rd-and-15 is treated identically to a 10-yard completion on 3rd-and-3, despite the vastly different utility of those outcomes.
\item Complexity: Unlike simple arithmetic formulas, DVOA requires complex play-by-play data parsing and subjective baseline determinations, making it inaccessible for general calculation without specialized databases.
\end{itemize}

\subsection{RIPPEN: A New Measure of Efficiency}

To address these limitations, we propose a new metric: the Rush Independent Passing Player Efficiency Number (RIPPEN). Unlike traditional ratings that rely on linear weights, RIPPEN utilizes a Bayesian model to simulate the distribution of play outcomes for a specific quarterback.

RIPPEN answers a fundamental question: In a simulated world where a team starts on their own 20-yard line and only performs pass plays, how many points would the quarterback be expected to score?

By simulating drive outcomes (Touchdown, Field Goal, or Turnover) based on a quarterback’s specific posterior distributions for completion, yardage, and interception rates, we derive a metric that is both statistically robust and intuitively interpretable. RIPPEN represents the average number of points a player contributes per ten possessions. This metric improves upon existing systems because it is updated dynamically to current NFL data, relies entirely on open-source data and code, and provides a clear, unit-based measure of performance (points per game) rather than an arbitrary index.

\section{Methodology}
The Rush Independent Passing Player Efficiency Number (RIPPEN) evaluates a quarterback's performance by isolating their contribution from external factors such as field position and rushing support. To achieve this, we employ a simulation-based approach rooted in Bayesian inference.

\subsection{Data Sources and Processing}

To ensure reproducibility and transparency, this study utilizes publicly available play-by-play data obtained via `nflreadr`. This package serves as the primary data access point for the "nflverse," a collection of open-source libraries dedicated to NFL analytics, and retrieves pre-processed data originally sourced from the National Football League (NFL).

The dataset includes comprehensive play-by-play observations for all NFL regular-season games. For the purposes of RIPPEN, we filter the data to isolate passing plays, specifically extracting four fundamental variables for each quarterback: pass attempts, completions, interceptions, and yards gained. This variable selection intentionally mirrors the inputs of the traditional NFL Passer Rating to demonstrate how the same simple inputs can yield a more robust metric through improved modeling.

A notable distinction in the `nflreadr` data is the granular breakdown of total yardage into "air yards" and "yards after catch" (YAC). While some metrics attempt to isolate quarterback performance by strictly measuring air yards, RIPPEN utilizes total passing yards. We argue that a quarterback's efficiency inherently includes the ability to identify and target receivers in advantageous positions to generate post-catch yardage; therefore, penalizing a quarterback for high YAC would remove a critical component of effective passing.

Finally, the data is processed to identify "censored" events. Touchdown passes are flagged not as exact yardage endpoints, but as lower-bound events (e.g., a 10-yard touchdown implies the potential for at least 10 yards), a distinction that is crucial for the survival analysis component of our yardage model.

\subsection{Simulation Framework}

The core metric is derived by simulating hypothetical offensive drives. In this framework, a quarterback is placed in a standardized context: starting at their own 20-yard line with a "1st and 10" situation. The drive proceeds solely through passing plays, with outcomes determined by the quarterback's specific posterior predictive distributions.

For each simulated drive $j$, the outcome $D_j$ is determined is recorded in points:

\begin{itemize}
	\item Touchdown: return 7. 
	\item Field Goal: return 3.  
	\item Turnover/Punt: return 0.  
\end{itemize}

The final RIPPEN score is calculated as the average number of points scored per ten possessions:
$$\text{RIPPEN} = 10 \times \frac{1}{n_s} \sum_{j=1}^{n_s} D_j$$

\subsection{Bayesian Models for Play Outcomes}
\subsubsection{Completions}
For a given quarterback over a given period of time, let $n_{comp}$ and $n_{att}$ be the number of completed passes and the number of attempted passes, respectively.  We then have the following Bayesian model for completion percentage: 

$$
n_{comp} \sim Binomial(n_{att}, p_{comp})
$$
with the following prior
$$
p_{comp} \sim \beta(\alpha_c,\beta_c)
$$.

This yields the following posterior distribution for $p_{comp}$: 
$$
p_{comp} |n_{comp}, n_{att} \sim Beta(\alpha_c + n_{comp}, \beta_c +n_{att}-n_{comp})
$$.  

\subsubsection{Interceptions}
Similarly, for a given quarterback over a given period of time, let $n_{int}$ and $n_{inc}$ be the number of intercepted passes and the number of incomplete passes (interceptions are considered incomplete passes), respectively.  We then have the following Bayesian model for interception percentage given an incomplete pass: 

$$
n_{int} \sim Binomial(n_{inc}, p_{int})
$$
with the following prior
$$
p_{int} \sim Beta(\alpha_i,\beta_i)
$$.
This yields the following posterior distribution for $p_{int}$: 
$$
p_{int} |n_{int},n_{inc} \sim Beta(\alpha_i + n_{int}, \beta_i +n_{inc} - n_i)
$$.  
It is important to note that $p_{int}$ is estimating the probability of an interception given that the pass was incomplete, not the interception rate across all attempted passes.  

In all simulations, we chose to use non-informative priors and set all hyperparameters of these models equal to 1.  Specifically, 

$$
\alpha_c = \beta_c = \alpha_i = \beta_i = 1
$$.  


 (Another possible idea here is to use empircal Bayes where these priors are based on league average completion and interception rates.)


\subsubsection{Model for yardage given completion}
Let $y_i$ be the yards gained on the $i$-th completed pass, $y_i^{\star} = log(y_i +1)$, and $TD_{i}$ is an indicator equal to 1 if the $i$-th completion is a touchdown and 0 otherwise.  Since we are trying to model yards given a completed pass, we consider touchdowns to be censoring events.  That is, if a player throws a 10 yard touchdown pass, we know that the play we at least ten yards.  This leads to the following likelihood function for modeling yards: 
$$
L(\mu, \sigma^2 | {\bf y}^{\star}) \propto \prod_{i=1}^n f(y_i^{\star}|\mu, \sigma^2)^{1-TD_i} S(y_i^{\star}|\mu, \sigma^2)^{TD_i}
$$
where $f$ is the pdf of a normal distribution and $S$ is the survival function of a normal distribution both with parameters $\mu$ and $\sigma^2$.  

For priors on $\mu$ amd $\sigma^2$, we use: 

$$
\mu \sim Normal(0,10^{6})
$$

$$
\sigma^2 \sim Uniform(0,100)
$$.  
\subsection{Simulation Algorithm}


\section{Results}


\subsection{Correlation between RIPPEN and winning}
Compare RIPPEN and winning to QBR and winning.  

\subsection{QBR vs RIPPEN Example. }

\subsection{Distribtuion of RIPPEN}

\subsection{Best Games/Seasons}

\section{Conclusion and Future Work}
RIPPEN is good.  We will do more eventually.  

Adding a defensive adjustment.  


Do we even want to add these things?  
How do we deal with pass interference?  
Defensive Holding? 
Sacks? Add another layer.  
Fumbles? Could treat similar to interceptions? 
Should interceptions ever result in negative numbers?  
How do we assign the negative numbers for interceptions?  
\section{Appendix}



\bibliographystyle{imsart-nameyear}
\bibliography{refs}

\href{http://bleacherreport.com/articles/889199-big-flaws-in-espns-total-qbr-exposed-after-tim-tebow-rates-above-aaron-rodgers}{Tim Tebow example of why QBR is bad:}\\

Read more about this \href{https://412sportsanalytics.wordpress.com/2016/10/05/pareto-frontier-and-non-dominated-quarterbacks/}{(pareto-frontier)}.  Might be interesting

- Would we add something like this to our results\\

\href{https://www.footballoutsiders.com/stats/qb}{DYAR and DVOA:}\\


\href{http://www.nih.ticz.musclehedz.charlespoliquin.sportsci.org/2011/mep.htm}{nih: charles poliquin}\\


\href{https://www.degruyter.com/view/j/jqas.2011.7.3/jqas.2011.7.3.1359/jqas.2011.7.3.1359.xml?format=INT}{JQAS}\\

\href{https://www.profootballhof.com/news/nfl-s-passer-rating/}{NFL Passer rating:}\\

\href{http://www.donsteinberg.com/qbrating.htm}{Don Steinberg: How I Learned to Stop Worrying and Love the Bomb}\\

\href{http://www.espn.com/blog/statsinfo/post/\_/id/123701/how-is-total-qbr-calculated-we-explain-our-quarterback-rating}{Total QBR calculation:}\\

http://www.math.montana.edu/graduate/writing-projects/2017/Gomez17.pdf

A Statistical Analysis of NFL Quarterback Rating Variables\\
Derek Stimel, Journal of Quantitative Analysis in Sports\\
\hfill\\
The Quarterback Prediction Problem: Forecasting the Performance of College Quarterbacks Selected in the NFL Draft\\
Julian Wolfson et al., Journal of Quantitative Analysis in Sports\\
\hfill\\
Analyzing dependence matrices to investigate relationships between national football 
league combine event performances\\
Brook T. Russell et al., Journal of Quantitative Analysis in Sports\\
\hfill\\
Isolating the Effect of Individual Linemen on the Passing Game in the National Football League\\
Benjamin C Alamar et al., Journal of Quantitative Analysis in Sports\\
\hfill\\
Quantifying NFL Coaching: A Proof of New Growth Theory\\
Kevin P. Braig, Journal of Quantitative Analysis in Sports\\

CITE Passer Rating\\
CITE QBR\\

\href{http://www.donsteinberg.com/qbrating.htm}{Don Steinberg: How I Learned to Stop Worrying and Love the Bomb}\\

\href{http://www.nfl.com/help/quarterbackratingformula}{Quarterback Rating:}\\

\href{https://www.profootballhof.com/news/nfl-s-passer-rating/}{NFL Passer rating:}\\

\href{http://football.stassen.com/pass-eff/}{College Passer efficiency:}\\

\href{https://www.si.com/more-sports/2011/08/03/defending-qb-rating}{Defending Passer rating: Kerry Byrne}\\

\href{https://www.nytimes.com/2004/01/14/sports/pro-football-the-nfl-s-passer-rating-arcane-and-misunderstood.html}{PRO FOOTBALL; The N.F.L.'s Passer Rating, Arcane and Misunderstood}\\

\subsubsection{Criticism of QBR}
Arbitrary scale (0 to 158.3??)
Hard to interpret (What does 121.6 mean?)
QBR overly credits QBs for scoring TDs -- discuss whether or not this is entirely wrong. Something to be said for "getting er done", but they weight this a bit too much for a metric which assesses QB efficacy.
\begin{quote}
[Don] Smith thought it would be more meaningful if an excellent score came to around 100, just like in school. "I think our attitude was that 100 was an A," he recalls. "And anything above 100, that was an A-plus." So, in a move that made sense at the time and has had everyone else confused for three decades, he multiplied the raw total by 100 and divided by 6, turning a statistically average performance -- 1s across the board -- into 66.7. It also made the maximum rating a ridiculous 158.2.
(http://www.donsteinberg.com/qbrating.htm)
\end{quote}

\end{document}
